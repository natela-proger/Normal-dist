\documentclass[a1paper, 14pt, portrait, 
               innermargin = 35mm, 
               blockverticalspace=0.25cm
               ]{tikzposter}
\usepackage{xltxtra}
\usepackage{polyglossia}
\setmainlanguage{russian}
\setotherlanguage{english}
\setkeys{russian}{babelshorthands=true}
\usepackage{fontspec}  
\usepackage{xcolor}

% TikzPoster Presets:

	\definecolor{storm}{HTML}{325D79}
	\definecolor{teal}{HTML}{9BD7D1}
	\definecolor{sunburst}{HTML}{F26627}
	\definecolor{sand}{HTML}{F9A26C}
	\definecolor{cloud}{HTML}{EFEEEE}
	
	\definecolor{weber}{HTML}{c29800}
	\definecolor{db}{HTML}{1c1309}
	\definecolor{bg}{HTML}{f2f1f0}

	\definecolorstyle{Normal}{
		\colorlet{One}{storm}
		\colorlet{Two}{teal}
		\colorlet{Three}{sunburst}
	}{
	% Background
		\colorlet{backgroundcolor}{white}
		\colorlet{framecolor}{storm}
	% Title
		\colorlet{titlefgcolor}{weber}
		\colorlet{titlebgcolor}{weber}
	% Blocks
		\colorlet{blocktitlebgcolor}{sand} 
		\colorlet{blocktitlefgcolor}{db} 
		\colorlet{blockbodybgcolor}{white} 
		\colorlet{blockbodyfgcolor}{db}
	% Innerblock colors 
		\colorlet{innerblocktitlebgcolor}{white} 	\colorlet{innerblocktitlefgcolor}{black} \colorlet{innerblockbodybgcolor}{white} \colorlet{innerblockbodyfgcolor}{black}
	% Note colors 
		\colorlet{notefgcolor}{black} 
		\colorlet{notebgcolor}{sunburst} 
		\colorlet{notefrcolor}{white}
	}


	\defineblockstyle{mtec}{
    titlewidthscale=1, bodywidthscale=1, titleleft,
    titleoffsetx=0pt, titleoffsety=0pt, bodyoffsetx=5pt, bodyoffsety=0pt,
    bodyverticalshift=0pt, roundedcorners=0, linewidth=0.2cm,
    titleinnersep=1cm, bodyinnersep=1cm
}{
    \begin{scope}[line width=\blocklinewidth, rounded corners=\blockroundedcorners]
       \ifBlockHasTitle %
           \draw[color=weber, line width = 3pt]
               ([xshift=30pt, yshift=5pt]blocktitle.south west) -- ([xshift=-30pt, yshift=5pt]blocktitle.south east);%
                \else
             \draw[draw=none]%, fill=blockbodybgcolor]
                 (blockbody.south west) rectangle (blockbody.north east);
        \fi
    \end{scope}
}  

%\usecolorstyle[colorPalette=GreenGrayViolet]{Denmark}
\usecolorstyle{Normal}
\useblockstyle{mtec}
\usetitlestyle{Empty}


% Font Presets:

    \usepackage{unicode-math}
      \setmathfont{Libertinus Math}
      \setmathfont[range=\mathup]{KazimirText-Book.ttf}
      \setmathfont[range=it/{latin, Latin, num}]{KazimirText-BookIt.ttf}
      %\setmathfont[range={"005B,"005D,"0028,"0029,"007B,"007D}]{KazimirText-BookIt.ttf}

    %\setmathfont{Fonts/RobotoSlab-SemiBold.ttf}
    
    \setromanfont[
            BoldFont=KazimirText-Bold.ttf, 
            ItalicFont = KazimirText-BookIt.ttf,
            FontFace = {l}{n}{KazimirText-Light.ttf},
            FontFace = {eb}{n}{KazimirText-Black.ttf}, 
            FontFace = {sb}{n}{KazimirText-SemiBold.ttf},
            Ligatures={Rare},
            Numbers={Proportional},
            SmallCapsFont={KazimirText-Regular.ttf},
            SmallCapsFeatures={Letters=SmallCaps}
            ]{KazimirText-Regular.ttf}
%    \newfontfamily\cyrillicfont{KazimirText-Book.ttf}
    
    \setsansfont[Ligatures=TeX, BoldFont=Jura-Bold.ttf, FontFace = {sb}{n}{Jura-Medium.ttf}]{Jura-SemiBold.ttf}
    
    \newfontfamily\kazimir{Kazimir-Regular.ttf}[%
        ItalicFont = Kazimir-Italic.ttf,
        BoldFont = Kazimir-Bold.ttf,
        BoldItalicFont = Kazimir-BoldItalic.ttf,
        FontFace = {l}{n}{Kazimir-Light.ttf},
        FontFace = {l}{it}{Kazimir-LightItalic.ttf},
        FontFace = {sb}{n}{Kazimir-Medium.ttf},
        FontFace = {sb}{it}{Kazimir-MediumItalic.ttf},
        FontFace = {eb}{n}{Kazimir-Black.ttf},
        FontFace = {eb}{it}{Kazimir-BlackItalic.ttf}
        ]
    \newfontfamily\jura{Jura-Regular.ttf}[%
        BoldFont = Jura-Bold.ttf,
        FontFace = {sb}{n}{Jura-Medium.ttf}
    ]
    \newfontfamily\roboto{RobotoCondensed-Regular.ttf}
    
    
    
    \tikzset{
       BaseFont/.style    = { font = \basefont },
       BrightFont/.style  = { BaseFont, white },
       NameFont/.style    = { BrightFont,
                              font = \basefont\fontsize{40pt}{48pt}\selectfont },
       TitleFont/.style   = { BrightFont,
                              font = \basefont\fontsize{16pt}{20pt}\selectfont },
    }


	\usepackage{xpatch}
	\xpatchcmd{\block}{\bf\LARGE\color{blocktitlefgcolor}}{\sffamily\LARGE\color{blocktitlefgcolor}}{}{} 


% Stuff:
	\usepackage{graphicx}
	\usepackage{tikz}
	\usepackage{pgfplots}
		\pgfplotsset{compat=1.15}
		\pgfmathdeclarefunction{gauss}{2}{%
  \pgfmathparse{1/(#2*sqrt(2*pi))*exp(-((x-#1)^2)/(2*#2^2))}%
}
	\usepackage{pgfplotstable}
	\usepackage{multicol}
        \setlength\columnsep{50pt}
	\usepackage{fontawesome5}
	\usepackage{etoolbox}
	\usepackage{soul}
	\usepackage[colorlinks=false]{hyperref}
	\usepackage{setspace}
        \setstretch{1.2}
    \usepackage{lettrine}
    \usepackage{booktabs}
    \renewcommand{\arraystretch}{1.2}
	 


% Turn off numeration of Tikz captions:
	\renewenvironment{tikzfigure}[1][]{
		\def \rememberparameter{#1}
		\vspace{10pt}
		\refstepcounter{figurecounter}
		\begin{center}
		}{
			\ifx\rememberparameter\@empty
			\else %nothing
			\\[10pt]
			%{\small Fig.~\thefigurecounter: \rememberparameter}
			{\small \rememberparameter}
			\fi
		\end{center}
	}


\title{\fontsize{50}{50}\selectfont \sffamily\bfseries Нормальное распределение}
\author{\color{db}
\fontsize{17}{17}\selectfont 
\fontspec{KazimirText-Regular.ttf}[Variant=1]Кордзахия, Никулина, Скворцов \& Татаринов \faUserAstronaut}
%\titlegraphic{\resizebox{\linewidth}{!}{\begin{tikzpicture}\begin{axis}[ticks=none,x post scale=6,every axis plot post/.append style={mark=none,domain=-20:20,samples=50,smooth},axis line style={draw=none},axis x line*=bottom,axis y line*=center,enlargelimits=false]\addplot {gauss(0,5)};\addplot{gauss(0,7)};\addplot{gauss(0,9)};\addplot{gauss(0,12)};\end{axis}\end{tikzpicture}}}
\date{\today}

\usepackage{blindtext}
\usetikzlibrary{positioning}



\begin{document}

\node[above right,opacity=0.1,inner sep=0pt,outer sep=0pt] at (-29.6,%33.5
-41) {\resizebox{\linewidth}{!}{\begin{tikzpicture}\begin{axis}[ticks=none,x post scale=0.6,every axis plot post/.append style={mark=none,domain=-20:20,samples=50,smooth},axis line style={draw=none},axis x line*=bottom,axis y line*=center,enlargelimits=false]\addplot[color = weber] {gauss(0,5)};\addplot[color = db]{gauss(0,7)};\addplot[color = sand]{gauss(0,9)};\addplot[color = teal]{gauss(0,12)};\end{axis}\end{tikzpicture}}};

\maketitle[
        %titletoblockverticalspace = -0.5cm
    ]

\block{Сквозь тернии к звездам}{
    \begin{multicols}{3}
        \lettrine{Н}{ормальное распределение} обладает самой красивой функцией плотности среди всех распределений:
            \begin{equation*}
                f\,(x) = \frac{1}{\sqrt{2\pi\sigma^2}}\; e^{-\frac{1}{2}\left(\frac{x-\mu}{\sigma}\right)^2}
            \end{equation*}
            
        Но почему оно выглядит именно так, а не иначе? 
        Вид нормального распределения — результат работы нескольких поколений ученых  {\scshape xvi-xix} столетий. Исток нормального распределения, как и исток многих человеческих достижений, кроется в человеческих ошибках — а точнее, в ошибках астрономических измерений.
        
        Астрономические исследования требуют точности и в то же время подвержены ошибкам наблюдения. Вопрос о том, как получать максимально точные оценки координат небесных тел, был актуален еще со времен Древней Греции. Во втором веке до н. э. Гиппарх использовал среднее минимума и максимума; четыре же столетия спустя Птолемей грешил выбором наиболее удобного для объяснения значения. Ученые эпохи Возрождения привыкли к множественному сбору данных, но не смогли определиться с конвертацией их в единый показатель: помимо среднего и медианы, использовались уникальные и не всегда очевидные методы. Например, основываясь на следующих наблюдениях:
        
        \vspace{1ex}
        \resizebox{\linewidth}{!}{
        \centering \selectfont\roboto
            \begin{tabular}{@{}lccc@{}}
            \multicolumn{4}{l}{
                Прямое восхождение Марса 13/23 января 1600 г. в 11 часов 50 минут:
            } \\
            \toprule
                & $\circ$ & $'$ & $''$ \\ \midrule
                используя яркую звезду стопы Близнецов & 134 & 23 & 39 \\
                используя Регул & 134 & 27 & 37 \\
                используя Поллукс & 134 & 23 & 18 \\
                в 12 ч. 17 м., используя третью звезду крыла Девы & 134 & 29 & 48 \\ \midrule
                Среднее с учетом однозначности наблюдений: & 134 & 24 & 33 \\ \bottomrule
            \end{tabular}
        }
        \vspace{1ex}
        
        Иоганн Кеплер получил собственную оценку прямого восхождения Марса — 134$^\circ$ 26$'$ 5.5$''$. До сих пор неизвестно, каким образом эта оценка была получена. 
        
        Как бы то ни было, астрономы того времени знали о существовании ошибок наблюдения и пытались с ними бороться.

        

    \end{multicols}
} 

\block{Гауссова кривая}{
\begin{multicols}{3}
    Впервые свойства случайных ошибок были описаны Галилео Галилеем в 1632 году, каждое из них  до сих пор используется в рамках статистики и эконометрики:
    \begin{itemize}
        \item  Существует одно, истинное значение наблюдаемой переменной.
        \item  Все наблюдения подвержены ошибкам, связанным с наблюдателем, инструментарием и прочими условиями.
        \item  Наблюдения распределены равномерно около истинного значения. Значит, ошибки распределены равномерно вокруг нуля.
        \item  Маленькие ошибки более вероятны, нежели большие.
    \end{itemize}
    
    Из утверждений Галилео следовало, что при достаточно большой выборке для оценки истинного значения достаточно посчитать среднее всех наблюдений, так как даже серьезные отклонения будут нивелированы своей малой частотой появления и не вызовут сильных искажений.
    
    Но как формализовать функцию таким образом, чтобы минимизировать вероятность сделанной ошибки? Ответ на этот вопрос был дан почти 200 лет спустя; в 1801 году итальянский астроном Джузеппе Пьяцци обнаружил небесное тело, ныне именуемое Церерой, и сделал предположение о ее планетарном характере. К сожалению, сделанных наблюдений не хватило, чтобы надежно вычислить орбиту Цереры: через полгода после обнаружения она скрылась за Солнцем. Перед астрономами встала задача вычислить место ее повторного появления, основываясь на 21 наблюдении, сделанном Пьяцци.
    
    Для вычисления места второго появления Цереры Гаусс использовал разработанный им метод наименьших квадратов, который утверждает, что лучшей оценкой некоего истинного параметра $\mu$ является среднее его наблюдений, поскольку  оно решает задачу минимизации квадратичной функции ошибок:
    
    \[
    \arg \min\limits_{\mu} \sum \epsilon_i^2 = \sum (y_i - \mu)^2 = \frac{y_1 + \ldots + y_n}{n} = \bar y
    \]
    Чтобы защитить этот подход, Гауссу необходимо было найти такое распределение ошибок, которое согласовывалось бы с предпосылками, сделанными Галилео, и обеспечивало бы максимальную правдивость имеющихся сведений при $\hat\mu = \bar y$. 

   Таким образом, {\fontseries{sb}\selectfont Гаусс делает следующие предположения:}
    \begin{itemize}
    	\item $\mu$ --- истинное значение;
    	\item $y_i = \mu + \epsilon_i$, где $\epsilon_i$ --- ошибка измерения;
    	\item $\epsilon_i$ --- н.о.р.с.в., симметрично распределенные относительно нуля;
    	\item Вероятность правдивости имеющихся сведений максимальна, если за истинное значение принимается среднее: $\arg\max\limits_{\mu} f_{Y_1, \ldots, Y_n}(y_1, \ldots, y_n; \mu) = \bar{y}$ .
    \end{itemize}
    
    Оказалось, что этим требованиям удовлетворяет именно колоколообразная кривая!
\end{multicols}


}


\begin{columns} 
    \column{1} \block{Доказательство}{
    \begin{multicols}{3}
    Решим задачу по поиску максимума функции плотности этого случайного вектора:
    \[ L = f_{Y_1,..., Y_n}(y_1, ..., y_n; \mu) = f_{Y_1}(y_1; \mu) \cdot f_{Y_2}(y_2; \mu) \cdot \ldots \cdot f_{Y_n}(y_n; \mu)\] 
    При этом заметим, что $f_{Y_i}(y_i; \mu) = f_{\epsilon_i}(y_i - \mu)$. Тогда 
    \[ L = f_{\epsilon}(y_1 - \mu) \times f_{\epsilon}(y_2 - \mu) \times ... \times f_{\epsilon}(y_n - \mu)\]
    Проведем монотонное преобразование:
    \[ \ln L = \ln f_{\epsilon}(y_1 - \mu) + ... +  \ln f_{\epsilon}(y_n - \mu) \rightarrow \underset{\mu}{max} \]
    \[ \frac{\partial \ln L}{\partial \mu} = \frac{f_{\epsilon}^{'}(y_1 - \mu)(-1)}{f_{\epsilon}(y_1 - \mu)} + ... + \frac{f_{\epsilon}^{'}(y_n - \mu)(-1)}{f_{\epsilon}(y_n - \mu)} = 0\]
    \[ \sum_i{\frac{f_{\epsilon}^{'}(y_i - \mu)}{f_{\epsilon}(y_i - \mu)}} = 0\]

    Теперь попробуем сделать некоторые выводы о распределении ошибок $\epsilon_i$, а, точнее,- об их функции плотности  $f_\epsilon(\cdot)$. \\
    
    Так как значения $y_i$ могут быть любыми, предположим, что, при общем количестве измерений равном $n + 1$:
     $y_1 = y_2 = .. = y_n = M$, причем $M \neq \bar{y}$ и $M - \bar{y} = \alpha$. Тогда
    
    \[ \bar{y} = \frac{ny_1 + y_{n + 1}}{n + 1}\]
    \[ y_{n+1} - \bar{y} = n(\bar{y} - y_1) = n\alpha\]
    
    Таким образом, мы можем представить сумму как
    \[ \Rightarrow \sum_i = n\frac{f_{\epsilon}^{'}(\alpha)}{f_{\epsilon}(\alpha)} + \frac{f_{\epsilon}^{'}(-n\alpha)}{f_{\epsilon}(-n\alpha)} = 0\] 
    
    \[ n\frac{f_{\epsilon}^{'}(\alpha)}{f_{\epsilon}(\alpha)} =  \frac{f_{\epsilon}^{'}(n\alpha)}{f_{\epsilon}(n\alpha)} \]
    
    Если обозначить $\frac{f_{\epsilon}^{'}(\alpha)}{f_{\epsilon}(\alpha)} = g(\alpha)$, то можно заметить, что функция $g(\alpha)$, удовлетворяющая условию $g(n\alpha) = ng(\alpha)$, - линейна. 
     
    Тогда представим $g(\alpha)$, как линейную функцию, зависящую от некоторого параметра $\gamma$
    
    \[  \frac{f_{\epsilon}^{'}(\alpha)}{f_{\epsilon}(\alpha)} =  \gamma \alpha\]
    \[ \frac{d f/ d \alpha}{f} = \gamma \alpha\]
    \[ \int \frac{d f}{f} = \int \gamma \alpha d \alpha\] 
    \[ \ln f = \frac{1}{2} \gamma \alpha^2 + C\] 
    \[ f = C \times \exp{(\dfrac{1}{2}\gamma \alpha^2)}, \gamma < 0, C > 0\]
    Пусть $\gamma = - \frac{1}{\sigma^2}$
    \[ f_\epsilon(x) = C \exp{(-\frac{x^2}{2\sigma^2})} \rightarrow \epsilon_i \sim N(0, \sigma^2)\]
    \[ \mu + \epsilon_i = Y_i \sim N(\mu, \sigma^2)\]
    
    \end{multicols}
    }
    
%    \column{0.4} \block{Почему <<нормальное>>?}{
%        Тут
%    }
\end{columns}


\end{document}
